\documentclass[twocolumn]{article}
\usepackage{mathtools}
\usepackage{listings}

\begin{document}
\title{ $\phi$-Calculus: A formal semantics of the silk programming language}
\author{Edward Dalley}
\maketitle

\begin{abstract}
Here we describe a formal semantics for the interpretation of a class programs as an executable dynamic data-flow graph.
This formal semantics is a specification for the behaviour of the silk programming language. As such the input axioms to the calculus are
the fundamental forms that are represented by the code which conforms to the silk language grammar.
\end{abstract}


\section{Introduction}

A silk program is a string of characters which can be reconised by the silk language grammar, which is descibed here.
These programs are composed of varous structures each deriving from a single fundamental expression type called a value.
We have the objective of converting these programs in a rigorous way into an executable structure
which has a well defined sequence of internal changes for each well defined external interaction.

\subsection{Phases}
There are several stages to the interpretation of silk programs. Formulation, Connection, Conduction and execution
\subsubsection{Encoding}
This is how the parsed forms are translated into a common form, primative forms can be formulated without the application of connection rules

A cell is a sequence of input contacts, a sequence of output contacts and a sequence of designations
\[
C : \langle I , O , D \rangle
\]

all contacts are distinct symbol values .

all designators are a target  \(\langle \hat{r}, \bar{c}, v? \rangle \)

we describe for each form the way that connections are created beteen entities within the context.

\begin{tabular}{ l | r }
  Syntactic Form && Semantic Form \\
  \italic{p} && \( C: \langle (\phi), (\phi), \Phi \rangle \) \\
\end{tabular}

\begin{lstlisting}
{a: b}
\end{lstlisting}

\section{Definitions}


\(
  \forall x \in X, \quad \exists y \leq \gamma
\)

\end{document}
